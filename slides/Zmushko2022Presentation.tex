\documentclass{beamer}
\beamertemplatenavigationsymbolsempty
\usecolortheme{beaver}
\setbeamertemplate{blocks}[rounded=true, shadow=true]
\setbeamertemplate{footline}[page number]
%
\usepackage[utf8]{inputenc}
\usepackage[english,russian]{babel}
\usepackage{amssymb,amsfonts,amsmath,mathtext}
\usepackage{subfig}
\usepackage[all]{xy} % xy package for diagrams
\usepackage{array}
\usepackage{multicol}% many columns in slide
\usepackage{hyperref}% urls
\usepackage{hhline}%tables
% Your figures are here:
\graphicspath{ {fig/} {../fig/} }

%----------------------------------------------------------------------------------------------------------

\begin{document}

%-----------------------------------------------------------------------------------------------------
\begin{frame}{Доклад с одним слайдом}



\begin{columns}[c]
\column{0.6\textwidth}
\includegraphics[width=1.0\textwidth]{data_visualization.png}

\column{0.4\textwidth}
\Large{\textbf{Neural ODE}}

\normalsize
\bigskip


$\mathbf{h_{t+1}} = \mathbf{h_t} + f(\mathbf{h_t}, \theta)$

\large

\bigskip


$\frac{d\mathbf{h}(t)}{dt} = f(\mathbf{h_t}, t, \theta)$

\normalsize
\bigskip 

$\mathbf{h}_{t_1} = \mathbf{h}_{t_0} + \int_{t_0}^{t_1} f(\mathbf{h}, t, \theta) dt$


\end{columns}
\begin{columns}[c]
\column{0.6\textwidth}
\includegraphics[width=1.0\textwidth]{cde_continuous_time.png}
\column{0.4\textwidth}
\large

\textbf{Цель}: декодирование сигнала моделями на основе Neural ODE.

\end{columns}

\end{frame}





\end{document} 