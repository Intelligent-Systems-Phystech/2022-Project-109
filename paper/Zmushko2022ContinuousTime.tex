\documentclass{article}
\usepackage{arxiv}

\usepackage[utf8]{inputenc}
\usepackage[english, russian]{babel}
\usepackage[T1]{fontenc}
\usepackage{url}
\usepackage{booktabs}
\usepackage{amsfonts}
\usepackage{nicefrac}
\usepackage{microtype}
\usepackage{lipsum}
\usepackage{graphicx}
\usepackage{natbib}
\usepackage{doi}


\title{Непрерывное время при построении нейроинтерфейса BCI}

\author{ Змушко Филипп \\
	МФТИ \\
	\texttt{zmushko.fa@phystech.edu} \\
	%% examples of more authors
	\And
	Самохина Алина \\
	МФТИ\\
	\texttt{alina.samokhina@phystech.edu} \\
	%% examples of more authors
	\And
	Стрижов Вадим \\
	МФТИ\\
}
\date{}

\renewcommand{\shorttitle}{\textit{arXiv} Template}

%%% Add PDF metadata to help others organize their library
%%% Once the PDF is generated, you can check the metadata with
%%% $ pdfinfo template.pdf
\hypersetup{
pdftitle={A template for the arxiv style},
pdfsubject={q-bio.NC, q-bio.QM},
pdfauthor={David S.~Hippocampus, Elias D.~Striatum},
pdfkeywords={First keyword, Second keyword, More},
}

\begin{document}
\maketitle

\begin{abstract}
	В задачах декодирования сигнала непрерывный процесс исследуется с помощью рекуррентных нейронных сетей (RNN), использующих дискретное представление о времени. Недавно появившиеся модели, основывающиеся на нейронных обыкновенных дифференциальных уравнений дают возможность рассматривать временные ряды как непрерывные во времени.
    В данной работе исследуется задача декодирования сигнала через представление в виде непрерывной во времени функции. Проводится сравнение базовых моделей реккуретных нейронных сетей (RNN, CNN) с моделями на основе нейронных дифференциальных уравнений - обыкновенных (NODE) и управляемых (NCDE). Исследуются возможности данных моделей как для регулярных сигналов, так и для сигналов с пропущенными значениями. 
\end{abstract}


\keywords{Нейронные ОДУ \and Управляемые ДУ \and  временные ряды \and классификация}

\section{Introduction}


\bibliographystyle{unsrtnat}
\bibliography{references}

\end{document}
