\documentclass{article}
\usepackage{arxiv}

\usepackage[]{cite}
\usepackage{cmap}
\usepackage{amsmath, amsfonts,amssymb}
\usepackage[utf8]{inputenc}
\usepackage[english, russian]{babel}
\usepackage[T1]{fontenc}
\usepackage{url}
\usepackage{booktabs}
\usepackage{amsfonts}
\usepackage{nicefrac}
\usepackage{microtype}
\usepackage{lipsum}
\usepackage{graphicx}
\usepackage{natbib}
\usepackage{doi}

\newcommand\argmin{\mathop{\arg\min}}
\newcommand{\T}{^{\text{\tiny\sffamily\upshape\mdseries T}}}
\newcommand{\hchi}{\hat{\boldsymbol{\chi}}}
\newcommand{\hphi}{\hat{\boldsymbol{\varphi}}}
\newcommand{\bchi}{\boldsymbol{\chi}}
\newcommand{\A}{\mathbf{A}}
\newcommand{\bb}{\mathbf{b}}
\newcommand{\B}{\mathcal{B}}
\newcommand{\W}{\mathbf{W}}
\newcommand{\E}{\mathbf{E}}
\newcommand{\x}{\mathbf{x}}
\newcommand{\y}{\mathbf{y}}
\newcommand{\Y}{\mathbf{Y}}
\newcommand{\X}{\mathbf{X}}
\newcommand{\Z}{\mathbf{Z}}
\newcommand{\hx}{\hat{x}}
\newcommand{\hX}{\hat{\X}}
\newcommand{\hy}{\hat{y}}
\newcommand{\M}{\mathcal{M}}
\newcommand{\N}{\mathcal{N}}
\newcommand{\R}{\mathbb{R}}
\newcommand{\p}{p(\cdot)}
\newcommand{\cc}{\mathbf{c}}
\newcommand{\m}{\mathbf{m}}
\newcommand{\bt}{\mathbf{t}}
\newcommand{\e}{\mathbf{e}}
\newcommand{\h}{\mathbf{h}}
\newcommand{\q}{q(\cdot)}
\newcommand{\uu}{\mathbf{u}}
\newcommand{\vv}{\mathbf{v}}
\newcommand{\dd}{\partial}

\renewcommand{\baselinestretch}{1}




\title{Непрерывное время при построении нейроинтерфейса BCI}

\author{ Змушко Филипп \\
	МФТИ \\
	\texttt{zmushko.fa@phystech.edu} \\
	%% examples of more authors
	\And
	Самохина Алина \\
	МФТИ\\
	\texttt{alina.samokhina@phystech.edu} \\
	%% examples of more authors
	\And
	Стрижов Вадим \\
	МФТИ\\
	\texttt{strijov@phystech.edu} \\
}
\date{}

\renewcommand{\shorttitle}{\textit{arXiv} Template}

%%% Add PDF metadata to help others organize their library
%%% Once the PDF is generated, you can check the metadata with
%%% $ pdfinfo template.pdf
\hypersetup{
pdftitle={A template for the arxiv style},
pdfsubject={q-bio.NC, q-bio.QM},
pdfauthor={David S.~Hippocampus, Elias D.~Striatum},
pdfkeywords={First keyword, Second keyword, More},
}

\begin{document}
\maketitle

\begin{abstract}
	В задачах декодирования сигнала непрерывный процесс исследуется с помощью рекуррентных нейронных сетей, использующих дискретное представление о времени. Недавно появившиеся модели, основывающиеся на нейронных обыкновенных дифференциальных уравнений рассматривают временные ряды как непрерывные во времени.
    В данной работе исследуется задача декодирования сигнала через представление в виде непрерывной во времени функции. Проводится сравнение базовых моделей, использующих дискретное время, с моделями на основе нейронных дифференциальных уравнений: обыкновенных и управляемых. Исследуется применение данных моделей как для регулярных сигналов, так и для сигналов с пропущенными значениями. 
\end{abstract}


\keywords{Нейронные ОДУ \and Управляемые ДУ \and  временные ряды \and классификация}

\section{Введение}
Данная статья посвящена  использованию непрерывного по времени сигнала при построении интерфейса мозг-компьютер (ИМК). ИМК~--- технология, позволяющая человеку взаимодействовать с компьютером посредством обработки данных об электрической активности мозга. В данной работе рассматривается задача классификации электроэнцефалограмм (ЭЭГ). Основной сложностью при работе с сигналами ЭЭГ является зашумленность и нерегулярный шаг по временной сетке. Это происходит из-за того, что обычно данные устройства предназначены для массового употребления, и, как следствие, имеют малое число датчиков и низкую частоту дискретизации сигнала. Для решения данной проблемы предлагается использовать модели на основе непрерывного представления сигнала во времени.

Большинство моделей такого типа основывается идее нейронных дифференциальных уравнений. Данная концепция базируется на схожести формулы для скрытого состояния блока ResNet с формулой Эйлера для решения обыкновенных дифференциальных уравнений (ODE). Эта идея, изложенная в \cite{NEURIPS2018_69386f6b}, рассматривает скрытое состояние ResNet как непрерывное во времени. Дальнейшее развитие этой концепции привело к появлению целого класса моделей, использующих непрерывное представление времени в рекуррентных нейронных сетях. В \cite{lechner2020longterm} рассматривается непрерывный аналог LSTM. Применение нейронных ОДУ к временным рядам с нерегулярным шагом по временной сетке представлено в \cite{cde}. В \cite{NEURIPS2019_952285b9} авторы рассматривают непрерывное представление во времени в виде разложения на полиномы Лежандра. Использование нейронных ОДУ в байесовских моделях исследуется в \cite{NEURIPS2019_99a40143}. 

В данной работе рассмотрено использование вышеперечисленных моделей в задаче декодирования сигнала. Изучены применения данных моделей для решения следующих задач:

\begin{itemize}
  \item построение нейроинтерфейса, работающего с непрерывным представлением о времени
  \item обработка нерегулярных во времени сигналов
  \item получение непрерывного представления сигнала
\end{itemize}

Сравнение моделей на основе NODE, а также базовых моделей (RNN, EEGNet\cite{Lawhern2018EEGNetAC}, ERPCov TS LR\cite{6046114}), использующих дискретное представление времени, производится в рамках задачи классификации на наборе данных потенциалов P300. Рассматриваемые здесь потенциалы P300~--- вызванные потенциалы, являющиеся специфическим откликом электрической активности мозга в ответ на внешний стимул.

Вышеперечисленные модели работают с непрерывным представлением временных рядов, но при этом не используют свойство непрерывности сигнала. В данной работе исследуется получение непрерывного представления сигнала, которое в дальнейшем может быть использовано при построении новых признаковых простанств. Эти представления могут рассматриваться как скрытые пространства, которые используются для согласовывания исходного сигнала с прогнозом с помощью корреляционного анализа.

\section{Постановка задачи}

\subsection{Задача классификации сигнала}
    
    Рассматривается задача классификации на выборке из регулярных по времени данных. Выборка делится на две части: обучающую и валидационную в соотношении 80:20.
    
    Пусть, в выборке присутствует $M$ наблюдений. Тогда сигнал и целевая переменная определяются следующим образом:
    

    $$\X = \{\X_i\}_{i=1}^{M},$$
    $$\X_i = \{\x_t\}_{t\in T}, \ \x_t \in \R^E, \ T = \{t_i\}_{i=1}^{N}, t_i \in \R$$ 
    
    $$\Y = \{y_i\}_{i=1}^{M},\ y_i \in \{0, 1\}$$
    
    $$E \text{~--- количество электродов}$$
    $$N \text{~--- количество наблюдений в одном отрезке ЭЭГ}$$

    \paragraph{Постановка задачи классификации.}
    
    В парадигме P300 решается задача бинарной классификации отрезков ЭЭГ. Задача ~--- определить наличие в отрезке ЭЭГ потенциала P300.
    
    Для данных, рассмотренных в двух предыдущих разделах, требуется получить целевую функцию:
    $$g_{\theta}: \X \to \Y.$$
    
    Критерием качества в данной задаче является бинарная кросс-энтропия: 
    $$L =  -{\frac {1}{M}}\sum _{m=1}^{M}\ {\bigg [}y_{m}\log {p}_{m}+(1-y_{m})\log(1-{p}_{m}){\bigg ]}$$
    $$p_m = g_{\theta}(\X_m) \text{ ~--- вероятность 1 класса для } \X_m$$

    Решается оптимизационная задача:
    \begin{equation*}
    \hat{\theta} = \arg\max_{\theta} L(\theta, \X).
    \end{equation*}
    
    Внешними критериями качества для задачи классификации являются точность и F1-score.
    
\subsection{Построение непрерывного представления сигнала}\\
    
    Еще одной задачей данной работы является получение непрерывного представления сигнала. 
    Пусть, имеется непрерывный процесс (активность мозга, движение):
    $$V(t), t \in \R$$
    Тогда данные выборки, регистрируемый сигнал ~--- реализация процесса $V(t)$:
    $$\X = \{\x_t\}_{t\in T},\  \x_t \in \R^E,\  T = \{t_i\}_{i=1}^{N}, t_i \in \R$$ 
    $$\x_{t_i} \approx V(t_i).$$
    Предполагается, что можно получить:
    $$f_{\X}(t): \R \to \R^E, \ f_{\X}(t) = \x_t \approx V(t).$$

    В данной работе построение непрерывного представления рассматривается в рамках решения задачи классификации сигналов. Непрерывное представление извлекается из скрытых состояний рассматриваемых далее алгоритмов классификации.



\bibliographystyle{plain}


\bibliography{Zmushko2022ContinuousTime}



\end{document}
